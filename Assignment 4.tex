\documentclass[10pt,a4paper]{report}
\usepackage[utf8]{inputenc}
\usepackage{amsmath}
\usepackage{amsfonts}
\usepackage{amssymb}


\begin{document}
\author{TARA AMBALE AWOII\\
		16/U/11837/EVE\\
		216018402
}
\title{REVIEW ON GOOGLE FIT AND WHY IT WAS CREATED}
\maketitle
	\begin{flushleft}
		\section{Introduction}
			Google Fit is a health-tracking platform developed by Google for the Android Operating System. It is a single set of APIs that blends data from multiple apps and devices. Google Fit uses trackers in a user's activity tracker or mobile device to record physical fitness activities such as walking or cycling, which are measured against the user's fitness goals to provide a comprehensive view of their fitness.\cite{1} \linebreak
			\linebreak
			Google fit app aggregates data from all your devices (smartphone, android wear smartwatch) and health fitness apps into one place. Users can check their fitness progress on the google fit website, tablets, smartphones. \linebreak
			 \linebreak
			 Google fit app is available as a free download\cite{2}\linebreak
			  \linebreak
			  \subsection{Pros and Cons of Google Fit}\cite{3}
			  \subsubsection{Cons}
			  	\begin{itemize}
			  	\item The interface is clean and pretty but not as much as the FitBit interface
			  	\item There is no social aspect for the app
			  	\item You can't log custom activities\linebreak
			  	\linebreak
			  	\subsubsection{Pros}
			  	\item The calendar view and color coded activity view on the web interface is a fun way to view the big picture of your activity levels.
			  	\item There are fitness challenges programmed into the watch app
			  	\item Google Fit seems to be compatible with a lot of apps.\linebreak
			  	\linebreak
			  	
		\section{Conclusion}
		In conclusion, Google fit is the answer to health and exercise tracking. It is very convenient, easy and fun to use.
		\section{References}
		\ref{Wikipedia} 
		\ref{Giffgaff community}
		\ref{little miss runshine}
		\bibliography{Assignment 4}
		\bibliographystyle{te} 
			  	
			  	\end{itemize}
	\end{flushleft}
\end{document}

